\documentclass[a4paper, 12pt]{report}
%================================================================
\usepackage{color,amsmath,amsfonts,amssymb,epsfig,hyperref}
\usepackage{graphicx}
\usepackage{dsfont}
\usepackage[latin1]{inputenc}
\usepackage[T1]{fontenc}
\usepackage[english]{babel}
\usepackage{pdfpages}
%====================================
 \textheight 24cm
% \doublespace
  \oddsidemargin -0.5cm
  \evensidemargin +1.5cm
  \textwidth 17cm
 \topmargin -2cm
%============================================
% Figures
%============================================
\newcommand{\figsstit}[2]{
\begin{figure}[hbtp]
\centerline{
    \hbox{ \includegraphics[scale=#2]{#1} }
}
\end{figure}}
%============================================
\newcommand{\figscale}[4]{
\begin{figure}[hbtp]
\centerline{
    \hbox{ \includegraphics[scale=#4]{#1} }
}
\begin{center}
\parbox{14 cm}
{
    \caption{\protect\small\it  {#2}}
    \label {#3}
}
\end{center}

\end{figure}}

%==================================================
\newcommand\algo[1]%
{
    \begin{center} %
    \begin{tabular} {||p{10 cm}l ||}%
    \hline
               #1 &  \\
    \hline
    \end{tabular}
    \vspace{12pt}
    \end{center}
}


%==============================================
\newcommand{\prob}[1]{\mathds{P}\left( #1 \right)}
\newcommand{\esp}[1]{\mathds{E}\left[ #1 \right]}
\newcommand{\var}[1]{\mathrm{var}\left( #1 \right)}
\newcommand{\cov}[1]{\mathrm{cov}\left( #1 \right)}
\newcommand{\diag}[1]{\mathrm{diag}\left( #1 \right)}
\newcommand{\trace}[1]{\mathrm{trace}\left( #1 \right)}
\newcommand{\card}[1]{\left| #1 \right|}
\newcommand{\myemph}[1]{\emph{\color{red}#1}}

%%============================================================================
\def\thesection{\arabic{section}}
%\def\thesubsection{\arabic{section}.\arabic{subsection}}
%\def\thesubsubsection{\arabic{section}.\arabic{subsection}.\arabic{subsubsection}}
%\def\thefigure{\arabic{figure}}
%\def\theequation{\arabic{equation}}
%\def\theexercice{\arabic{exercice}}
%\def\theexample{\arabic{example}}
%\def\theproof{\arabic{proof}}

%===============================================
\newtheorem{property}{Properties}
\newtheorem{remark}{Remark}
\newtheorem{theorem}{Theorem}
\newtheorem{definition}{Definition}
\newtheorem{example}{Example}
\newtheorem{lemme}{Lemme - \thelemme}
\newtheorem{proof}{Proof - \theproof}
\newenvironment{TAB}{\begin{table}[[hbt] \center \leavevmode}{\end{table}}
%%============================================================================
%\renewcommand\arraystretch{1.6}

\def\ua{\underline a}
\def\ub{\underline b}
\def\uB{\underline B}
\def\uH{\underline H}
\def\ur{\underline r}
\def\us{\underline s}
\def\ux{\underline x}
\def\uX{\underline X}
\def\uZ{\underline Z}
\def\utheta{\underline \theta}



\def\tn{\mathrm{TN}}
\def\fn{\mathrm{FN}}
\def\tp{\mathrm{TP}}
\def\fp{\mathrm{FP}}
\def\tpn{\mathrm{tPN}}
\def\tnn{\mathrm{tNN}}
\def\tdn{\mathrm{tDN}}

\def\precision{\mathrm{\color{red}Precision}}
\def\recall{\mathrm{\color{red}Recall}}
\def\fscore{{\color{red}F\mathrm{-score}}}
\def\far{\mathrm{FAR}}
\def\mdr{\mathrm{MDR}}
\def\vdr{\mathrm{VDR}}
\def\ci{\mathrm{CI}}
\def\pfa{P_{\mathrm{FA}}}
\def\pd{P_{\mathrm{D}}}
\def\loc{\mathrm{LOC}}

\def\SNR{\mathrm{SNR}}
\def\crb{\mathrm{CRB}}
\def\fim{\mathrm{FIM}}

\def\auc{\mathrm{AUC}}
\def\aec{\mathrm{aec}}



\def\void{{\small void}}
\def\nomeaning{{\small meaningless}}
\def\unknown{{\small unknown}}
\def\MSC{\mathrm{MSC}}
\def\hMSC{\widehat{\MSC}}%{\MSC}} 
\def\ellk{{k}}
\def\SOI{common signal part }
\def\absGamma{\Phi}

%============== colors ========================
\definecolor{enstrouge}{RGB}{212,65,84}
\definecolor{lightorange}{RGB}{235,226,52}
\definecolor{greennoise}{RGB}{243,42,255}
\definecolor{lightred}{RGB}{255,181,183}
\definecolor{light-grey}{rgb}{0.95,0.95,0.95}
\definecolor{peach}{rgb}{0.98,0.49,0.25}
\definecolor{burntorange}{rgb}{0.79,0.37,0}
\definecolor{light-yellow}{rgb}{1,1,0.92}

\definecolor{light-green}{RGB}{231,255,145}
\definecolor{enstorange}{RGB}{255,214,10}
\definecolor{enstrouge}{RGB}{212,65,84}
\definecolor{grey}{RGB}{204,204,204}
\definecolor{blue}{RGB}{0,0,255}
\definecolor{almost-black}{RGB}{100,100,100}
\definecolor{violet}{rgb}{0.4,0,0.4}
\definecolor{cyan}{RGB}{0,255,255}
\definecolor{magenta}{RGB}{243,42,255}

\def\degree{^{\circ}}
\def\simiid{\stackrel{\mathrm{i.i.d.}}{\sim}}
\def\simind{\stackrel{\mathrm{ind.}}{\sim}}

 
%%%============================================================================
%%\def\thesection{\arabic{section}}
%%\def\thefigure{\arabic{figure}}
%%\def\theequation{\arabic{equation}}
%%\def\theexercice{\arabic{exercice}}
%%\def\theequation{\arabic{exercice}.\arabic{equation}}
%%%============================================================================
%%\newcounter{auxiliaire}
%%%%%%%%% comment
%%\setcounter{auxiliaire}{\theenumi}
%%\end{enumerate}
%% TEXTE
%%\begin{enumerate}
%%\setcounter{enumi}{\theauxiliaire}
%%%============================================================================

 \bibliographystyle{plain} 

\begin{document}
 \sloppy
%=======================================================
%=======================================================
\section{Introduction}
A 3D seismic sensor  consists of 3 sensors, each one with a direction of measurement. Each direction is associated to the Theoretically it is required that the 3 directions are orthogonal.

The signal depends of the dot product between $u_{k}$ and the DOA of the source, assumed to be faraway  in such a way the wave can be considered as plane. Usually, by construction, the 3 vectors $u_{k}$ are orthogonal.

If we considered 2 seismic sensors it is not sure that the 3 arm directions coincide. If not the transformation that brings one sensor to the other is a 3 D rotation.


We consider a 3D sensor denoted SUT and a 3D sensor denoted SREF. We denote $x_{u}(t)$ the 3D time series of the SUT and $x_{r}(t)$ the 3D time series of the SREF.  Therefore we can write in the frequency domain:

\begin{eqnarray*}
\left\{
\begin{array}{rcl}
X_{u}(f)&=&R(\theta) H_{u}(f)S(f)
\\
X_{r}(f)&=&H_{r}(f)S(f)
\end{array}\right .
\end{eqnarray*}
where $R$ is a rotation matrix, which in the most general case depends the 3D parameter $\theta$, where $H_{u}(f)$ and $H_{r}(f)$  are the 3D frequency responses of each arm of the respective sensors.
It follows that
\begin{eqnarray*}
X_{u}(f) &=&R(\theta)H_{u}(f) (H_{r}^{T}(f)H_{r}(f))^{-1}H_{r}^{T}(f)X(f)
\end{eqnarray*}
Then
\begin{eqnarray*}
X_{u}(f)X_{u}^{T}(f) &=&R(\theta)H_{u}(f) (H_{r}^{T}(f)H_{r}(f))^{-1}H_{r}^{T}(f)X_{r}(f)X_{u}^{T}(f)
\end{eqnarray*}
and
\begin{eqnarray*}
X_{u}(f)X_{u}^{T}(f) &=&R(\theta)H_{u}(f) (H_{r}^{T}(f)H_{r}(f))^{-1}H_{r}^{T}(f)X_{r}(f)X_{u}^{T}(f)
\end{eqnarray*}

\begin{eqnarray*}
X_{u}(f)X_{u}^{T}(f) &=&R(\theta)a(f)H_{u}(f) X_{u}^{T}(f)
\end{eqnarray*}
where $a(f)=(H_{r}^{T}(f)H_{r}(f))^{-1}H_{r}^{T}(f)X_{r}(f)$ is a scalar. Let $S(f)=X_{u}(f)X_{u}^{T}(f)$ and $Q(f)=H_{u}(f) X_{u}^{T}(f)$. We have
\begin{eqnarray*}
S(f)&=&R(\theta)a(f)Q(f)
\end{eqnarray*}
By integration on $f$, we have:
\begin{eqnarray*}
S&=&R(\theta) P
\end{eqnarray*}
Usually $S$ and $P$ are full rank, therefore $R$ can be derived from $S$ and $P$ but with the constraint that $R$ is unitary.
A way could be to numerically minimize 
\begin{eqnarray*}
J(\theta) &=& \| S - R(\theta)P\|_{F}^{2}
\end{eqnarray*}
where $R(\theta)=R_{x}(\theta_{x})R_{y}(\theta_{y})R_{z}(\theta_{z})$ and
\begin{eqnarray*}
\begin{array}{ccc}
R_{x}(\theta_{x})=
\begin{bmatrix}
1&0&0\\
0 &\cos(\theta_{x})&-\sin(\theta_{x})\\
0&\sin(\theta_{x})&\cos(\theta_{x})
\end{bmatrix}
&
%\end{eqnarray*}
%
%\begin{eqnarray*}
R_{y}(\theta_{y})=
\begin{bmatrix}
\cos(\theta_{y})&0&\sin(\theta_{y})\\
0&1&0\\
-\sin(\theta_{y})&0&\cos(\theta_{y})
\end{bmatrix}
\\ \\
%\end{eqnarray*}
%
%\begin{eqnarray*}
R_{z}(\theta_{z})=
\begin{bmatrix}
\cos(\theta_{z})&-\sin(\theta_{z})&0\\
\sin(\theta_{z})&\cos(\theta_{z})&0\\
0&0&1
\end{bmatrix}
\end{array}
\end{eqnarray*}

\end{document}

